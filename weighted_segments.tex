\section{FPT for weighted segments with $\delta$-extensions}

TODO: Some intro

\begin{tw}[FPT for weighted segment cover with $\delta$-extensions]{
	\label{fpt_weighted_segment}
	There exists an algorithm that given a family $\sets$ of
	$n$ weighted segments (in any direction),
	a set of $m$ points $\points$, and parameters $k$ and $\delta > 0$,
	runs in time $f(k, \delta) \cdot (nm)^c$ for some computable function $f$ and a constant $c$,
	and outputs a set $\sol \subseteq \sets$
	such that $|\sol| \le k$ and $\sol^{+\delta}$ covers all points in $\points$,
	or determines that such a set $\sol$ does not exist.
}\end{tw}


To solve this problem we will introduce a lemma about choosing
a \textit{good} subset of points.

TODO: Some intuition

\begin{defi}
	For a set of collinear points $C$,
	a subset $A \subseteq C$ is \textbf{$(k,\delta)$-good} 
	if for any set of segments $R$ that covers $A$ and
	such that $|R| \le k$, it holds that $R^{+\delta}$ covers $C$.
\end{defi}

\begin{lemma}
	\label{good_set_exists}
	There exists an algorithm that
	for any set of collinear points $C$, $\delta > 0$ and $k \ge 1$,
	outputs a $(k,\delta)$-good set $A \subseteq C$ of size at most $f(k, \delta)$
	for some computable function $f$. This algorithm runs in time
	$O(|C| \cdot f(k, \delta))$.
\end{lemma}

\begin{proof}
We prove this for a fixed $\delta$ by induction over $k$.

\subparagraph{Inductive hypothesis.}
For any set of collinear points $C$,
there exists an algorithm that 
runs in time $O(|C|k(1+\frac{1}{\delta}))$
and finds a set $A$ such that:
\begin{itemize}
\item $A$ is $(\ell, \delta)$-good for every $1 \le \ell \le k$,
\item $A$ has size $|A| < f(\delta, k)$ for some computable function $f$,
\item extreme points from $C$ are in $A$.
\end{itemize}

\subparagraph{Base case for $k = 1$.}
It is sufficient that $A$ consists of 2 points: extreme points from $C$
or a single point if $|C| = 1$.

If they are covered with one segment, it must be a segment 
that includes the extreme points from $C$, so it covers the whole set $C$.

\subparagraph{Inductive step.}
Assuming inductive hypothesis for any set of collinear points $C$
and for parameter $k$, we will prove hypothesis for $k+1$.

Let be $s$ the minimal segment that includes all points from $C$.
That is, the extreme points of $C$ are endpoints of $s$.

We define $M = \lceil1+\frac{2}{\delta}\rceil$ subsegments of $s$ in the following way.
We split $s$ into $M$ parts 
$v_i$ of equal length, that is $|v_i| = \frac{|s|}{M}$ for each $1 \le i \le M$.

Let $C_i$ be the subset of $C$ consisting of points laying on $v_i$.

Let $t_i$ be the segment with endpoints being the extreme points of $C_i$
(it might be degenerated segment if $|C_i| = 1$ or it might be empty
if $C_i$ is empty).

TODO: Add a picture with $v_i$ and $t_i$ here

We use the inductive hypothesis to choose $(k, \delta)$-good sets $A_i$
for sets $C_i$. Note that if $|C_i| \le 1$, then $A_i = C_i$
and it's still a $(k, \delta)$-good set for $C_i$.

Then we define $A = \bigcup_{i=1}^{M} A_i$.
Thus $A$ includes the extreme points of $C$,
because they are included in the sets $A_1$ and $A_M$.

\subparagraph{Proof that $A$ is $(k, \delta)$-good for $C$.}
Let us take any cover of $A$ with $k+1$ segments and call it $\sol$.

For every segment $t_i$, if there exists a segment $x$ in $\sol$ 
that is disjoint with $t_i$,
then we have a cover of $A_i$ with at most $k$
segments using $\sol - \{x\}$.
Since $A_i$ is $(k, \delta)$-good for $t_i$ and $C_i$,
then $(\sol - \{x\})^{+\delta}$ covers $C_i$.
So $\sol^{+\delta}$ covers $C_i$ as well.

If there exists a segment $t_i$ for which a segment $x$ as defined above
does not exist, then all $k+1$ segments that cover
$A_i$ intersect with $t_i$. (Note: There may exist only one such segment $t_i$).
From the inductive hypothesis end points of $s$ are
in $A_1$ and $A_M$ respectively, so $\sol$ must cover them.
Hence there must exist
segments starting in the ends of $s$ and ending somewhere in $t_i$.
Let us call these two segments $y$ and $z$. It follows that:
$|y| + |z| + |t_i| \ge |s|$.
Since $|t_i| \le |v_i| = \frac{|s|}{M} \le \frac{|s|}{1+\frac{2}{\delta}} = \frac{|s|\delta}{\delta+2}$,
we have $\max(|y|, |z|) \ge |s|(1-\frac{\delta}{\delta+2})/2 = \frac{|s|}{\delta+2}$.

TODO: Add a picture with such segments here

After $\delta$-extension, the longer of these segments will
lengthen both ways by at least:
$$\frac{|s|\delta}{\delta+2} = \frac{|s|}{1+\frac{2}{\delta}} > \frac{|s|}{M} = v_i > t_i.$$

Therefore the longer of segments $y$ and $z$ will cover the segment $t_i$
after $\delta$-extension, therefore $\sol^{+\delta}$ covers $C_i$.

Since $C = \bigcup_{i=1}^M C_i$,
then $\sol^{+\delta}$ covers $C$.

\subparagraph{Complexity}

We use the recursive algorithm for subsets $C_i$. Every point
from $C$ belongs to at most 2 sets $C_i$.

Apart from recursive algorithm we perform operations linear in
size of $|C| + M$ to calculate the sets $C_i$.

Therefore it has complexity:
$$O(|C|+M) + \sum_i^M O(|C_i|k(1+\frac{1}{\delta})) = 
O(|C| + (1+\frac{1}{\delta})) + O((\sum_i^M |C_i|)k(1+\frac{1}{\delta})) \le O(|C|k(1+\frac{1}{\delta})).$$

\end{proof}

\begin{proof}[Proof of Theorem \ref{fpt_weighted_segment}]

To construct an algorithm for this problem let us formulate
some claims about the problem first.

\begin{defi}
Line is \textbf{long} if there are at least $k+1$ points from $\points$ on it.
\end{defi}

\begin{claim}
If there are more than $k$ long lines, then 
$\points$ can not be covered with $k$ segments.
\end{claim}

\begin{claim}
If there is more than $k^2$ points from $\points$
that do not lie on any long line, then $\points$ can not be covered with $k$ segments.
\end{claim}

Applying the above claims, if we have more than $k$ long lines
or more than $k^2$ points form $\points$
that do not lie on any long line, then we answer that
there is no solution of size at most $k$.

Otherwise, we can split $\points$ into at most $k+1$ sets:
$D$, at most $k^2$ points that do not lie on any long line
and $C_i$ -- points
that lay on $i$-th long line.
Sets $C_i$ do not need to be disjoint.

Then for every set $C_i$, we can use Lemma \ref{good_set_exists}
to get $(k,\delta)$-good set $A_i$
for $C_i$.

Then we have set $D \cup \bigcup A_i$ of size at most $f(k, \delta)$
for some computable function $f$, that if we have a solution $\sol$ of size at most $k$
that covers $D \cup \bigcup A_i$, then $\sol^{+\delta}$ covers $\points$.
This is because
$\sol$ already covers points $D$, they cover $C_i$, because
they cover $(k,\delta)$-good set $A_i$ with at most $k$ segments,
so $\sol^{+\delta}$ covers $C_i$.

After that we shrunk down size of $\points$ to size of $f(k, \delta)$
for some computable function $f$.
Then we would like to shrink down size of $\sets$.
For every collinear subset of $D$, we can choose one segment from
$\sets$ that covers these points and have the lowest weight
or decide there is no segment that cover them.
There are at most $|D|^2$ different segments, because
we can distnguish these collinear sets by their extreme points.

This has complexity $O(|D|^2|\sets|)$ and produce shrunk down
set $\sets$ of size $f(k, \delta)$ for some computable function $f$.

Then we can iterate over all subsets of shrunk down set $\sets$ and
choose the set with the lowest sum of weights that cover $D$. This solution
would have weight not larger than optimal solution
for the problem without extension, because we iterate
over all posibilities of covering the subset of $\points$.

\end{proof}

