\section{FPT for weighted segments with $\delta$-extensions}
\begin{tw}[FPT for weighted segment cover with $\delta$-extensions]{
	\label{fpt_weighted_segment}
	There exists an algorithm $\mathcal{A}$ that given a family $\sets$ of
	$n$ weighted segments (in any direction),
	a set of $m$ points $\points$, and parameters $k$ and $\delta$,
	runs in time $f(k, \delta) \cdot (nm)^c$ for some computable function $f$ and a constant $c$,
	and outputs a set $\sol \subseteq \sets$
	such that $|\sol| \le k$ and $\sol^{+\delta}$ covers all points in $\points$
	or determines that such a set $\sol$ does not exist.
}\end{tw}


To solve this problem we will introduce a lemma about choosing
\textit{good} subsets of points.

\begin{defi}
	For a set of colinear points $C$,
	a subset $A \subseteq C$ is \textbf{$(k,\delta)$-good} 
	if for any set of segments $R$ that covers set $A$
	such that $|R| \le k$, it holds that $R^{+\delta}$ covers $C$.
\end{defi}

\begin{lemma}
	\label{good_set_exists}
	There exists an algorithm that
	for any set of colinear points $C$, $\delta > 0$ and $k \ge 1$,
	outputs a $(k,\delta)$-good set of size at most $f(k, \delta)$
	for some computable function $f$. This algorithm runs in time
	$O(|C| \cdot f(k, \delta))$.
\end{lemma}

\begin{proof}
We prove this for a fixed $\delta$
by induction over $k$ for any set of colinear points $C$.

\subparagraph{Inductive hypothesis}
For any set of colinear points $C$,
there exists an algorithm that 
runs in time $O(|C|k(1+\frac{1}{\delta}))$
and finds a set $A$ such that:
\begin{itemize}
\item $A$ is $(l, \delta)-good$ for every $1 \le l \le k$,
\item $A$ has size $|A| < f(\delta, k)$ for some computable function $f$,
\item extreme points from $C$ are in $A$.
\end{itemize}

\subparagraph{Base case for $k = 1$}
It is sufficient that $A$ consists of 2 points: extreme points from $C$
or a single point if $|C| = 1$.

If they are covered with one segment, it must be a segment 
that includes the extreme points from $C$, so it covers whole set $C$.

\subparagraph{Inductive step}
Assuming inductive hypothesis for any set of colinear points $C$
and for $k$, we will prove hypothesis for $k+1$.

Let us name $s$ the minimal segment that includes all points from $C$.

We define $M = \lceil1+\frac{2}{\delta}\rceil$ subsegments of $s$ in the following way.
We split $s$ into $M$ parts 
$v_i$ of equal length, that is $|v_i| = \frac{|s|}{M}$ for any $1 \le i \le M$.

$C_i$ is a subset of $C$ such that they lay on $v_i$.

$t_i$ is a segment connecting leftmost and rightmost point in $C_i$
(it might be degenerated segment if $|C_i| = 1$ or it might be empty
if $C_i$ is empty).

TODO: Add a picture with $v_i$ and $t_i$ here

We use inductive hypothesis to choose $(k, \delta)$-good sets $A_i$
for sets $C_i$. If $|C_i| \le 1$, then $A_i = C_i$
and it's still a $(k, \delta)$-good set.

Then we define $A = \bigcup_{i=1}^{M} A_i$.
It includes ends of $s$, because they are in sets $A_1$ and $A_M$.

\subparagraph{Proof that $A$ is $(k, \delta)$-good for $C$}
Let us take any cover of $A$ with $k+1$ segments and name it $\sol$.

For every segment $t_i$, if there exists a segment $x$ from $\sol$ 
such that it is disjoint with $t_i$,
then we have a cover of $A_i$ with at most $k$
segments using $\sol - \{x\}$.
Since $A_i$ is $(k, \delta)$-good for $t_i$ and $C_i$,
then $(\sol - \{x\})^{+\delta}$ covers $C_i$.

If there exists a segment $t_i$ for which a segment $x$ as defined above
does not exist, then all $k+1$ segments that cover
$A_i$ intersect with $t_i$. (Note: There exists only one such segment $t_i$).
From the inductive hypothesis ends of $s$ are
in $A_1$ and $A_M$ respectively, so $\sol$ must cover them.
Hence there must exist
segments starting in the ends of $s$ and ending somewhere in $t_i$.
Let us name these two segments $y$ and $z$. It follows that:
$|y| + |z| + |t_i| \ge |s|$.
Since $|t_i| \le |v_i| = \frac{|s|}{M} \le \frac{|s|}{1+\frac{2}{\delta}} = \frac{|s|\delta}{\delta+2}$,
therefore $max(|y|, |z|) > |s|(1-\frac{\delta}{\delta+2})/2 = \frac{|s|}{\delta+2}$.

TODO: Add a picture with such segments here

After $\delta$-extension, the longer of these segments will
lengthen both ways by at least:
$$\frac{|s|\delta}{\delta+2} = \frac{|s|}{1+\frac{2}{\delta}} > \frac{|s|}{M} = v_i > t_i.$$

Therefore the longer of segments $y$ and $z$ will cover the segment $t_i$
after $\delta$-extension, therefore $\sol^{+\delta}$ covers $C_i$.

Since $C = \bigcup_{i=1}^M C_i$,
then $\sol^{+\delta}$ covers $C$.

\subparagraph{Complexity}

We use the recursive algorithm for subsets $C_i$. Every point
from $C$ belongs to at most 2 sets $C_i$.

Apart from recursive algorithm we perform operations linear in
size of $|C| + M$ to calculate the sets $C_i$.

Therefore it has complexity:
$$O(|C|+M) + \sum_i^M O(|C_i|k(1+\frac{1}{\delta})) = 
O(|C| + (1+\frac{1}{\delta})) + O((\sum_i^M |C_i|)k(1+\frac{1}{\delta})) \le O(|C|k(1+\frac{1}{\delta})).$$

\end{proof}

\begin{proof}[Proof of Theorem \ref{fpt_weighted_segment}]

To construct an algorithm for this problem let us formulate
some claims about the problem first.

\begin{defi}
Line is \textbf{long} if there are at least $k+1$ points from $\points$ on it.
\end{defi}

\begin{claim}
If there are more than $k$ long lines, then 
$\points$ can not be covered with $k$ segments.
\end{claim}

\begin{claim}
If there is more than $k^2$ points from $\points$
that do not lie on any long line, then $\points$ can not be covered with $k$ segments.
\end{claim}

Applying the above claims, if we have more than $k$ long lines
or more than $k^2$ points form $\points$
that do not lie on any long line, then we answer that
there is no solution of size at most $k$.

Otherwise, we can split $\points$ into at most $k+1$ sets:
$D$, at most $k^2$ points that do not lie on any long line
and $C_i$ -- points
that lay on $i$-th long line.
Sets $C_i$ do not need to be disjoint.

Then for every set $C_i$, we can use Lemma \ref{good_set_exists}
to get $(k,\delta)$-good set $A_i$
for $C_i$.

Then we have set $D \cup \bigcup A_i$ of size at most $f(k, \delta)$
for some computable function $f$, that if we have a solution $\sol$ of size at most $k$
that covers $D \cup \bigcup A_i$, then $\sol^{+\delta}$ covers $\points$.
This is because
$\sol$ already covers points $D$, they cover $C_i$, because
they cover $(k,\delta)$-good set $A_i$ with at most $k$ segments,
so $\sol^{+\delta}$ covers $C_i$.

After that we shrunk down size of $\points$ to size of $f(k, \delta)$
for some computable function $f$.
Then we would like to shrink down size of $\sets$.
For every colinear subset of $D$, we can choose one segment from
$\sets$ that covers these points and have the lowest weight
or decide there is no segment that cover them.
There are at most $|D|^2$ different segments, because
we can distnguish these colinear sets by their extreme points.

This has complexity $O(|D|^2|\sets|)$ and produce shrunk down
set $\sets$ of size $f(k, \delta)$ for some computable function $f$.

Then we can iterate over all subsets of shrunk down set $\sets$ and
choose the set with the lowest sum of weights that cover $D$. This solution
would have weight not larger than optimal solution
for the problem without extension, because we iterate
over all posibilities of covering the subset of $\points$.

\end{proof}

\section{W[1]-completeness for weighted segments in 3 directions}

\begin{tw}
	\textbf{W[1]-completeness for weighted segments in 3 directions}.
	Consider the problem of covering a set $\points$ of points
	by selecting $k$ axis-pararell or right-diagonal weighted segments
	with weights
	from a set $\sets$ with minimal weight.
	Assuming ETH, there is no algorithm for this
	problem with running time
	$f(k)\cdot(|\points| + |\sets|)^{o(\sqrt(k))}$
	for any computable function $f$.
\end{tw}

We will show reduction from grid tiling problem.


Let's have an instance of grid tiling problem -- size of the
gird $k$, number of elements available $n$
and $k^2$ sets of available pairs in every tile
$S_{i, j} \subseteq \{1,n\} \times \{1,n\}$.

\paragraph{Construction.}
We construct a set $\sets$ of segments and a set $\points$
of points.

First let's choose any ordering of $n^2$ elements
$\{1,n\} \times \{1,n\}$ and name this sequence $a_1 \ldots a_{n^2}$.

$$match_v(i, j) \iff
a_i = \{x_i, y_i\} \land a_j = \{x_j, y_j\} \land x_i = x_j$$

$$match_h(i, j) \iff
a_i = \{x_i, y_i\} \land a_j = \{x_j, y_j\} \land y_i = y_j$$


\subparagraph{Points.}

Define points:
	$$h_{i, j, t} = (j \cdot (n^2+1) + t, (n^2+1) \cdot i)$$
	$$v_{i, j, t} = ((n^2+1) \cdot i, j \cdot (n^2+1) + t)$$
	
Let's define sets $H$ and $V$ as:
$$H = \{h_{i, j, t} : 1 \le i, j, \le k, 1 \le t \le n^2\}$$
$$V = \{v_{i, j, t} : 1 \le i, j, \le k, 1 \le t \le n^2\}$$
	
Let's define $\epsilon = 0.1$.
For a point $\{x, y\} = p$ we define points
$p^{L} = \{x - \epsilon, y\}$,
$p^{R} = \{x + \epsilon, y\}$,
$p^{U} = \{x, y - \epsilon\}$,
and $p^{D} = \{x, y + \epsilon\}$.

Then we define:
$$\points := H \cup \{p^L : p \in H\} \cup \{p^R : p \in H\}
\cup V \cup \{p^U : p \in V\} \cup \{p^D : p \in V\} $$


\subparagraph{Segments.}
Define horizontal segments.

$$hor_{i, j, t_1, t_2} = (h^R_{i,j,t_1}, h^L_{i, j+1, t_2})$$
$$ver_{i, j, t_1, t_2} = (v^D_{i,j,t_1}, v^U_{i, j+1, t_2})$$

$$horbeg_{i, t} = (h^L_{i, 1, 1}, h^L_{i, 1, t})$$
$$horend_{i, t} = (h^R_{i, n, t}, h^R_{i, n, n^2})$$


$$verbeg_{i, t} = (v^U_{i, 1, 1}, v^U_{i, 1, t})$$
$$verend_{i, t} = (v^D_{i, n, t}, v^D_{i, n, n^2})$$

\begin{eqnarray*}
HOR &= &\{hor_{i, j, t_1, t_2} : 1 \le i \le k, 1 \le j < k,
1 \le t_1, t_2 \le n^2, match_h(t_1, t_2)\} \\
&\cup &\{horbeg_{i,t} : 1 \le i \le k, 1 \le t \le n^2\}
\\
&\cup &\{horend_{i,t} : 1 \le i \le k, 1 \le t \le n^2\}
\end{eqnarray*}

\begin{eqnarray*}
VER &= &\{ver_{i, j, t_1, t_2} : 1 \le i \le k, 1 \le j < k,
1 \le t_1, t_2 \le n^2, match_v(t_1, t_2)\} \\
&\cup &\{verbeg_{i,t} : 1 \le i \le k, 1 \le t \le n^2\}
\\
&\cup &\{verend_{i,t} : 1 \le i \le k, 1 \le t \le n^2\}
\end{eqnarray*}

$$DIAG := \{ (h_{i, j, t}, v_{j, i, t}) :
	1 \le i, j \le k, 1 \le t \le n^2, a_t \in S_{i, j}\}$$

TODO: explain that these segments are in fact diagonal

$$\sets := HOR \cup VER \cup DIAG$$



\begin{lemma}
	If there exists solution for grid tiling,
	then there exists solution for our construction
	using $2(k+1)k + k^2$ segments
	with weight exactly $2k \cdot (k(n^2+1) - 2 - 2\epsilon(k-1))$.
\end{lemma}

\begin{claim}
	If there exists a solution to the grid tiling
	$c_1 \ldots c_k$ and $r_1 \ldots r_k$,
	then there exists a solution covering
	all points
	$$\{h_{i, j, t} : 1 \le i, j \le k, t=(c_i, r_j)\}
	\cup \{v_{i, j, t} : 1 \le i, j \le k, t=(c_j, r_i)\}$$
	
	with segments in $DIAG$
	and the rest in $VER$ or $HOR$
	and has weight $2k \cdot (k(n^2+1) - 2 - 2\epsilon(k-1))$.
\end{claim}

\paragraph{Proof.}
TODO: jakiś prosty z definicji

\begin{lemma}
	If there exists solution for our construction
	using $2(k+1)k + k^2$ segments
	with weight exactly $2k \cdot (k(n^2+1) - 2 - 2\epsilon(k-1))$,
	then there exists a solution for grid tiling
\end{lemma}

\paragraph{Proof.}
This follows from Lemma $\ref{main_soundness_construction}$,
because we just take which points are covered with $DIAG$.

\begin{claim}
\label{guards}
Points $p^{L}, p^R, p^U, p^D$ cannot be covered with $DIAG$.
\end{claim}

\begin{claim}
\label{hor_ver_sound}
Points in $H \cup \{p^L : p \in H\} \cup \{p^R : p \in H\}$
cannot be covered with $VER$.

Points in $V \cup \{p^U : p \in V\} \cup \{p^D : p \in V\} $
cannot be covered with $HOR$.
\end{claim}

\begin{claim}
For given $i, j$ if none of the points $h_{i, j, t}$ ($v_{i, j, t}$)
for $1 \le t \le n^2$ are covered with $DIAG$,
then some spaces between neighbouring points were covered twice.
\end{claim}

\begin{claim}
For given $i, j$ two points $h_{i, j, t_1}, h_{i, j, t_2}$
($v_{i, j, t_1}, v_{i, j, t_2}$)
for $1 \le t_1 < t_2 \le n^2$ are covered with $DIAG$,
then one of them had to be also covered with a segment from $HOR$
($VER$).
\end{claim}
\paragraph{Proof.}
Point $v^L_{i, j, t_2}$ had to be covered with $VER$
from Claims $\ref{guards}$ and $\ref{hor_ver_sound}$.
And every segment in $VER$ covering $v^L_{i, j, t_2}$,
covers also $v^L_{i, j, t_1}$.

\begin{lemma}
	\label{main_soundness_construction}
	If there exists solution for our construction
	with weight at most (exactly)
	$2k \cdot (k(n^2+1) - 2 - 2\epsilon(k-1))$,
	then for every $i, j$
	there must be exactly one $t$ such that $h_{i, j, t}$
	($v_{i, j, t}$) 
	is covered with $DIAG$
	and moreover if $h_{i, j, t_1}$ and $h_{i, j+1, t_2}$
	are uncovered, then $math_h(t_1, t_2)$.
	Analogically for $v$.
\end{lemma}
\paragraph{Proof.}
Only $k^2$ points can be covered only in $DIAG$, the rest
has to be covered with $VER \cup HOR$.
Therefore every result must be at least $ALL\_LINES$ - $2k^2\epsilon$,
because only $2k^2$ spaces of length $\epsilon$
can be uncovered in this axis.

Of course if $h_{i, j, t_1}$ and $h_{i, j+1, t_2}$
are uncovered, then there must exist
a segment in $HOR$ between $h^R_{i, j, t_1}$ and $h^L_{i, j+1, t_2}$,
so $math_h(t_1, t_2)$ must be true.



\section{What is missing}
We don't know FPT for axis-pararell segments without $\delta$-extensions.
