\section{Weighted segments}
\subsection{FPT for weighted segments with $\delta$-extensions}
\begin{tw}{
	\textbf{(FPT for weighted segment cover with $\delta$-extensions)}.
	There exists an algorithm that given a family $\sets$ of
	$n$ weighted segments (in any direction),
	a set of $m$ points $\points$
	and a parameter $k$,
	runs in time $f(k) \cdot (nm)^c$ for some computable function $f$ and constant $c$,
	and outputs a subfamily $\sol \subseteq \sets$
	such that $|\sol| \le k$ and $\sol^{+\delta}$ covers all points in $\points$.
}\end{tw}

To solve this problem we will introduce kernel for slightly different
problem: Weighted segment cover of points and segments.
In shortcut: WSCPS.

\begin{lemma}
	\textbf{(Algorithm for kernel of WSCPS)}.
	There exists an algorithm that given a family $\sets$ of
	$n$ weighted segments (in any direction),
	a set of $m_1$ points $\points_1$ and $m_2$ segments $\points_2$
	and a parameter $k$,
	runs in time $f(k) \cdot g(m_1, m_2) \cdot n^c$ for some computable functions $f, g$ and constant $c$,
	and outputs a subfamily $sol \subseteq \sets$
	such that $|\sol| \le k$ and $\sol$ covers all points in $\points_1$ and all segments in $\points_2$. 
\end{lemma}
\paragraph{Proof}
Only sketch for now.

We can compute dynamic programming $dp(A, B, z)$ --
the best cost to cover at least whole segment $A, B$
using at most $z$ segments.
$A, B$ are all interesting points -- ends of any segment given on the input or points given on the input.
We can compute it in polynomial time.

Then we can create a new double weighted set (original weight,
number of used segments from $\sets$) -- $\sets_2$
that has only segments which never cover partially any segment from $\points_2$
(covers the whole segment or doesn't cover at all).
In such $\sets_2$ we can find solution $\sol$ where any 2 segments
have empty intersection (don't cover each other and don't meet at the ends).
Because if we had such solution, we can merge these two segments and
such segment there's also in $\sets_2$.

In that case we can find kernel of $\sets_2$ of size $k\cdot(m_1 + 2m_2)^2$,
because we only need to take the best weight covering some subset of $\points_1 \cup \points_2$.

\begin{lemma}
	\textbf{Kernel in WSCPS}.
	TODO: formulate it properly
	
	For segment cover, there is a kernel of size $f(k)$
	in WSCPS.
\end{lemma}
\paragraph{Proof.}
TODO

\begin{lemma}
	\label{covered_spaces_lemma}
	If all the points are covered with $k$ segments and
	the biggest $2(1 + 1/\delta)^{k+1}$ spaces between points
	are filled, the whole segment is filled after
	$\delta$-extensions of these segments.
\end{lemma}
\paragraph{Proof.}
	Let's name the $2(1 + 1/\delta)^{k+1}$-st biggest space
	between points as $y$.
	We have guarantee that all segements of length $x > y$
	are covered without $\delta$-extensions.
	
	Let's take one space between points that is not covered
	before $\delta$-extension and we will prove
	it will be covered after $\delta$-extensions.
	Let's assume it isn't.
	
	This space has length $x$. Since it's uncovered, $x \le y$.
	
	Let's take side where the sum of lengths of segments
	covering the points is greater (left or right).
	Without loss of generality, let us assume it's right.
	
	There are at most $k$ segments to the right of this space
	between points. Name their lengths $l_1, l_2 \ldots l_k$.
	If the point is covered in the other direction,
	the segment is degenerated to the point and $l_i = 0$.
	Name the space between endpoints of $l_i$ and $l_{i+1}$ -- $x_i$.
	Of course, $x_i$ is uncovered space between two points,
	therefore $x_i \le y$.
	
	TUTAJ BEDZIE PEWNIE RYSUNEK Z TYMI SUPER RZECZAMI DO PRZERW
	
	Let's write equations meaning that $i$-th segment
	doesn't cover space $x$ after $\delta$-expansion.
	
	$$l_1\delta < x \le y \then l_1 < y/\delta$$
	$$l_2\delta < x + l_1 + x_1 < 2y + y/\delta \then l_2 < 2y/\delta + y/\delta^2$$
	$$l_3\delta < x + l_1 + x_1 + l_2 + x_2 < 3y + 3y/\delta + y/\delta^2 \then l_3 < 3y/\delta +  3y/\delta^2 + y/\delta^3$$
	
	From this we can "guess" induction $l_i < y((1+1/\delta)^i - 1)$
	
	Trivailly for $l_1 < y/\delta$.
	
	Assume that for all $j < i$: $$l_j < y((1+1/\delta)^j - 1)$$.
	
	$l_i\delta < x + \sum_{j = 1}^{i-1}(l_j + x_j)
	< iy \sum_{j = 1}^{i-1}l_j
	< iy + \sum{j=1}^{i-1}y((1+1/\delta)^j - 1)
	= iy - (i-1)y + \sum{j=1}^{i-1}y(1+1/\delta)^j
	= y(1 + \sum_{j = 1}^{i-1}(1+1/\delta)^j)
	= y(2 + \sum_{j = 1}^{i-1}(1+1/\delta)^j - 1)
	= y(\sum_{j = 0}^{i-1}(1+1/\delta)^j - 1)
	= y((1+1/\delta)^i / (1 - (1+1/\delta)) - 1)
	= y((1+1/\delta)^i\delta - 1)
	< y((1+1/\delta)^i\delta - \delta)$
	
	Of course we also know that (since we have chosen the side with greater sum of the width of segments):
	$$\sum_{i=1}^{k} l_i \ge 1/2 \cdot y \cdot 2(1 + 1/\delta)^{k+1} =  y \cdot (1 + 1/\delta)^{k+1}$$
	
	But 
	$\sum_{i=1}^{k} l_i
	< \sum_{i=1}^{k} y((1+1/\delta)^i - 1)
	= y((1+1/\delta)^{k+1} / (1-(1+1/\delta)) - k)
	= y((1+1/\delta)^{k+1}\delta - k)
	< y(1+1/\delta)^{k+1}$
	
	Therefore the space must have been covered after $\delta$-expansions.

\subsection{W[1]-completeness for weighted segments in 3 directions}

\subsection{What is missing}
We don't know FPT for axis-pararell segments without $\delta$-extensions.
