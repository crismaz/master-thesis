\section{Weighted segments}
\subsection{FPT for weighted segments with $\delta$-extensions}
\begin{tw}{
	\textbf{(FPT for weighted segment cover with $\delta$-extensions)}.
	There exists an algorithm that given a family $\sets$ of
	$n$ weighted segments (in any direction),
	a set of $m$ points $\points$
	and a parameter $k$,
	runs in time $f(k) \cdot (nm)^c$ for some computable function $f$ and constant $c$,
	and outputs a subfamily $\sol \subseteq \sets$
	such that $|\sol| \le k$ and $\sol^{+\delta}$ covers all points in $\points$.
}\end{tw}

To solve this problem we will introduce kernel for slightly different
problem: Weighted segment cover of points and segments.
In shortcut: WSCPS.

\begin{lemma}
	\textbf{(Algorithm for kernel of WSCPS)}.
	There exists an algorithm that given a family $\sets$ of
	$n$ weighted segments (in any direction),
	a set of $m_1$ points $\points_1$ and $m_2$ segments $\points_2$
	and a parameter $k$,
	runs in time $f(k) \cdot g(m_1, m_2) \cdot n^c$ for some computable functions $f, g$ and constant $c$,
	and outputs a subfamily $sol \subseteq \sets$
	such that $|\sol| \le k$ and $\sol$ covers all points in $\points_1$ and all segments in $\points_2$. 
\end{lemma}
\paragraph{Proof}
Only sketch for now.

We can compute dynamic programming $dp(A, B, z)$ --
the best cost to cover at least whole segment $A, B$
using at most $z$ segments.
$A, B$ are all interesting points -- ends of any segment given on the input or points given on the input.
We can compute it in polynomial time.

Then we can create a new double weighted set (original weight,
number of used segments from $\sets$) -- $\sets_2$
that has only segments which never cover partially any segment from $\points_2$
(covers the whole segment or doesn't cover at all).
In such $\sets_2$ we can find solution $\sol$ where any 2 segments
have empty intersection (don't cover each other and don't meet at the ends).
Because if we had such solution, we can merge these two segments and
such segment there's also in $\sets_2$.

In that case we can find kernel of $\sets_2$ of size $k\cdot(m_1 + 2m_2)^2$,
because we only need to take the best weight covering some subset of $\points_1 \cup \points_2$.

\begin{lemma}
	\label{kernel_wscps}
	\textbf{Kernel in WSCPS}.
	TODO: formulate it properly
	
	For segment cover, there is a kernel of size $f(k)$
	in WSCPS.
\end{lemma}

\begin{claim}
If there are more than $k$ lines with
at least $k+1$ points on them, then 
they can't be covered with $k$ segments.
\end{claim}

\begin{claim}
If there is more than $k^2$ points
that don't lie on any line with more than $k$
points on it, then they can't be covered with $k$ segments.
\end{claim}

\begin{claim}
For every long line $L$ (with more than $k$ points on them)
we can choose $f(k)$ points on them,
that if we cover all of these points with at most $k$
segments, then the rest of the points
with $\delta$-extensions will be covered by
segments in the direction of line $L$.
\end{claim}


\paragraph{Proof of Lemma \ref{kernel_wscps}.}
After applying the previous lemmas,
we have at most $k^2 + k\cdot f(k)$
points that can be covered in any direction
and for the rest of the points we can draw 
at most $k\cdot f(k)$ segments along
their respective long lines that have to be covered
by segments after $\delta$-extensions.

Then we extend every available segment by $\delta$-extension
and we achieve the kernel in WSCPS for this instance of problem.


\begin{lemma}
	\label{covered_spaces_lemma}
	If all the points are covered with $k$ segments and
	the biggest $2(1 + 1/\delta)^{k+1}$ spaces between points
	are filled, the whole segment is filled after
	$\delta$-extensions of these segments.
\end{lemma}
\paragraph{Proof.}
	Let's name the $2(1 + 1/\delta)^{k+1}$-st biggest space
	between points as $y$.
	We have guarantee that all segements of length $x > y$
	are covered without $\delta$-extensions.
	
	Let's take one space between points that is not covered
	before $\delta$-extension and we will prove
	it will be covered after $\delta$-extensions.
	Let's assume it isn't.
	
	This space has length $x$. Since it's uncovered, $x \le y$.
	
	Let's take side where the sum of lengths of segments
	covering the points is greater (left or right).
	Without loss of generality, let us assume it's right.
	
	There are at most $k$ segments to the right of this space
	between points. Name their lengths $l_1, l_2 \ldots l_k$.
	If the point is covered in the other direction,
	the segment is degenerated to the point and $l_i = 0$.
	Name the space between endpoints of $l_i$ and $l_{i+1}$ -- $x_i$.
	Of course, $x_i$ is uncovered space between two points,
	therefore $x_i \le y$.
	
	TUTAJ BEDZIE PEWNIE RYSUNEK Z TYMI SUPER RZECZAMI DO PRZERW
	
	Let's write equations meaning that $i$-th segment
	doesn't cover space $x$ after $\delta$-expansion.
	
	$$l_1\delta < x \le y \then l_1 < y/\delta$$
	$$l_2\delta < x + l_1 + x_1 < 2y + y/\delta \then l_2 < 2y/\delta + y/\delta^2$$
	$$l_3\delta < x + l_1 + x_1 + l_2 + x_2 < 3y + 3y/\delta + y/\delta^2 \then l_3 < 3y/\delta +  3y/\delta^2 + y/\delta^3$$
	
	From this we can "guess" induction $l_i < y((1+1/\delta)^i - 1)$
	
	Trivailly for $l_1 < y/\delta$.
	
	Assume that for all $j < i$: $$l_j < y((1+1/\delta)^j - 1)$$.
	
	$l_i\delta < x + \sum_{j = 1}^{i-1}(l_j + x_j)
	< iy \sum_{j = 1}^{i-1}l_j
	< iy + \sum{j=1}^{i-1}y((1+1/\delta)^j - 1)
	= iy - (i-1)y + \sum{j=1}^{i-1}y(1+1/\delta)^j
	= y(1 + \sum_{j = 1}^{i-1}(1+1/\delta)^j)
	= y(2 + \sum_{j = 1}^{i-1}(1+1/\delta)^j - 1)
	= y(\sum_{j = 0}^{i-1}(1+1/\delta)^j - 1)
	= y((1+1/\delta)^i / (1 - (1+1/\delta)) - 1)
	= y((1+1/\delta)^i\delta - 1)
	< y((1+1/\delta)^i\delta - \delta)$
	
	Of course we also know that (since we have chosen the side with greater sum of the width of segments):
	$$\sum_{i=1}^{k} l_i \ge 1/2 \cdot y \cdot 2(1 + 1/\delta)^{k+1} =  y \cdot (1 + 1/\delta)^{k+1}$$
	
	But 
	$\sum_{i=1}^{k} l_i
	< \sum_{i=1}^{k} y((1+1/\delta)^i - 1)
	= y((1+1/\delta)^{k+1} / (1-(1+1/\delta)) - k)
	= y((1+1/\delta)^{k+1}\delta - k)
	< y(1+1/\delta)^{k+1}$
	
	Therefore the space must have been covered after $\delta$-expansions.

\subsection{W[1]-completeness for weighted segments in 3 directions}

\begin{tw}
	\textbf{W[1]-completeness for weighted segments in 3 directions}.
	Consider the problem of covering a set $\points$ of points
	by selecting $k$ axis-pararell or right-diagonal weighted segments
	with weights
	from a set $\sets$ with minimal weight.
	Assuming ETH, there is no algorithm for this
	problem with running time
	$f(k)\cdot(|\points| + |\sets|)^{o(\sqrt(k))}$
	for any computable function $f$.
\end{tw}

We will show reduction from grid tiling problem.


Let's have an instance of grid tiling problem -- size of the
gird $k$, number of elements available $n$
and $k^2$ sets of available pairs in every tile
$S_{i, j} \subseteq \{1,n\} \times \{1,n\}$.

\paragraph{Construction.}
We construct a set $\sets$ of segments and a set $\points$
of points.

First let's choose any ordering of $n^2$ elements
$\{1,n\} \times \{1,n\}$ and name this sequence $a_1 \ldots a_{n^2}$.

$$match_v(i, j) \iff
a_i = \{x_i, y_i\} \land a_j = \{x_j, y_j\} \land x_i = x_j$$

$$match_h(i, j) \iff
a_i = \{x_i, y_i\} \land a_j = \{x_j, y_j\} \land y_i = y_j$$


\subparagraph{Points.}

Define points:
	$$h_{i, j, t} = (j \cdot (n^2+1) + t, (n^2+1) \cdot i)$$
	$$v_{i, j, t} = ((n^2+1) \cdot i, j \cdot (n^2+1) + t)$$
	
Let's define sets $H$ and $V$ as:
$$H = \{h_{i, j, t} : 1 \le i, j, \le k, 1 \le t \le n^2\}$$
$$V = \{v_{i, j, t} : 1 \le i, j, \le k, 1 \le t \le n^2\}$$
	
Let's define $\epsilon = 0.1$.
For a point $\{x, y\} = p$ we define points
$p^{L} = \{x - \epsilon, y\}$,
$p^{R} = \{x + \epsilon, y\}$,
$p^{U} = \{x, y - \epsilon\}$,
and $p^{D} = \{x, y + \epsilon\}$.

Then we define:
$$\points := H \cup \{p^L : p \in H\} \cup \{p^R : p \in H\}
\cup V \cup \{p^U : p \in V\} \cup \{p^D : p \in V\} $$


\subparagraph{Segments.}
Define horizontal segments.

$$hor_{i, j, t_1, t_2} = (h^R_{i,j,t_1}, h^L_{i, j+1, t_2})$$
$$ver_{i, j, t_1, t_2} = (v^D_{i,j,t_1}, v^U_{i, j+1, t_2})$$

$$horbeg_{i, t} = (h^L_{i, 1, 1}, h^L_{i, 1, t})$$
$$horend_{i, t} = (h^R_{i, n, t}, h^R_{i, n, n^2})$$


$$verbeg_{i, t} = (v^U_{i, 1, 1}, v^U_{i, 1, t})$$
$$verend_{i, t} = (v^D_{i, n, t}, v^D_{i, n, n^2})$$

\begin{eqnarray*}
HOR &= &\{hor_{i, j, t_1, t_2} : 1 \le i \le k, 1 \le j < k,
1 \le t_1, t_2 \le n^2, match_h(t_1, t_2)\} \\
&\cup &\{horbeg_{i,t} : 1 \le i \le k, 1 \le t \le n^2\}
\\
&\cup &\{horend_{i,t} : 1 \le i \le k, 1 \le t \le n^2\}
\end{eqnarray*}

\begin{eqnarray*}
VER &= &\{ver_{i, j, t_1, t_2} : 1 \le i \le k, 1 \le j < k,
1 \le t_1, t_2 \le n^2, match_v(t_1, t_2)\} \\
&\cup &\{verbeg_{i,t} : 1 \le i \le k, 1 \le t \le n^2\}
\\
&\cup &\{verend_{i,t} : 1 \le i \le k, 1 \le t \le n^2\}
\end{eqnarray*}

$$DIAG := \{ (h_{i, j, t}, v_{j, i, t}) :
	1 \le i, j \le k, 1 \le t \le n^2, a_t \in S_{i, j}\}$$

TODO: explain that these segments are in fact diagonal

$$\sets := HOR \cup VER \cup DIAG$$



\begin{lemma}
	If there exists solution for grid tiling,
	then there exists solution for our construction
	using $2(k+1)k + k^2$ segments
	with weight exactly $2k \cdot (k(n^2+1) - 2 - 2\epsilon(k-1))$.
\end{lemma}

\begin{claim}
	If there exists a solution to the grid tiling
	$c_1 \ldots c_k$ and $r_1 \ldots r_k$,
	then there exists a solution covering
	all points
	$$\{h_{i, j, t} : 1 \le i, j \le k, t=(c_i, r_j)\}
	\cup \{v_{i, j, t} : 1 \le i, j \le k, t=(c_j, r_i)\}$$
	
	with segments in $DIAG$
	and the rest in $VER$ or $HOR$
	and has weight $2k \cdot (k(n^2+1) - 2 - 2\epsilon(k-1))$.
\end{claim}

\paragraph{Proof.}
TODO: jakiś prosty z definicji

\begin{lemma}
	If there exists solution for our construction
	using $2(k+1)k + k^2$ segments
	with weight exactly $2k \cdot (k(n^2+1) - 2 - 2\epsilon(k-1))$,
	then there exists a solution for grid tiling
\end{lemma}

\paragraph{Proof.}
This follows from Lemma $\ref{main_soundness_construction}$,
because we just take which points are covered with $DIAG$.

\begin{claim}
\label{guards}
Points $p^{L}, p^R, p^U, p^D$ cannot be covered with $DIAG$.
\end{claim}

\begin{claim}
\label{hor_ver_sound}
Points in $H \cup \{p^L : p \in H\} \cup \{p^R : p \in H\}$
cannot be covered with $VER$.

Points in $V \cup \{p^U : p \in V\} \cup \{p^D : p \in V\} $
cannot be covered with $HOR$.
\end{claim}

\begin{claim}
For given $i, j$ if none of the points $h_{i, j, t}$ ($v_{i, j, t}$)
for $1 \le t \le n^2$ are covered with $DIAG$,
then some spaces between neighbouring points were covered twice.
\end{claim}

\begin{claim}
For given $i, j$ two points $h_{i, j, t_1}, h_{i, j, t_2}$
($v_{i, j, t_1}, v_{i, j, t_2}$)
for $1 \le t_1 < t_2 \le n^2$ are covered with $DIAG$,
then one of them had to be also covered with a segment from $HOR$
($VER$).
\end{claim}
\paragraph{Proof.}
Point $v^L_{i, j, t_2}$ had to be covered with $VER$
from Claims $\ref{guards}$ and $\ref{hor_ver_sound}$.
And every segment in $VER$ covering $v^L_{i, j, t_2}$,
covers also $v^L_{i, j, t_1}$.

\begin{lemma}
	\label{main_soundness_construction}
	If there exists solution for our construction
	with weight at most (exactly)
	$2k \cdot (k(n^2+1) - 2 - 2\epsilon(k-1))$,
	then for every $i, j$
	there must be exactly one $t$ such that $h_{i, j, t}$
	($v_{i, j, t}$) 
	is covered with $DIAG$
	and moreover if $h_{i, j, t_1}$ and $h_{i, j+1, t_2}$
	are uncovered, then $math_h(t_1, t_2)$.
	Analogically for $v$.
\end{lemma}
\paragraph{Proof.}
Only $k^2$ points can be covered only in $DIAG$, the rest
has to be covered with $VER \cup HOR$.
Therefore every result must be at least $ALL\_LINES$ - $2k^2\epsilon$,
because only $2k^2$ spaces of length $\epsilon$
can be uncovered in this axis.

Of course if $h_{i, j, t_1}$ and $h_{i, j+1, t_2}$
are uncovered, then there must exist
a segment in $HOR$ between $h^R_{i, j, t_1}$ and $h^L_{i, j+1, t_2}$,
so $math_h(t_1, t_2)$ must be true.



\subsection{What is missing}
We don't know FPT for axis-pararell segments without $\delta$-extensions.
