%
% Niniejszy plik stanowi przykład formatowania pracy magisterskiej na
% Wydziale MIM UW.  Szkielet użytych poleceń można wykorzystywać do
% woli, np. formatujac wlasna prace.
%
% Zawartosc merytoryczna stanowi oryginalnosiagniecie
% naukowosciowe Marcina Wolinskiego.  Wszelkie prawa zastrzeżone.
%
% Copyright (c) 2001 by Marcin Woliński <M.Wolinski@gust.org.pl>
% Poprawki spowodowane zmianami przepisów - Marcin Szczuka, 1.10.2004
% Poprawki spowodowane zmianami przepisow i ujednolicenie 
% - Seweryn Karłowicz, 05.05.2006
% Dodanie wielu autorów i tłumaczenia na angielski - Kuba Pochrybniak, 29.11.2016

% dodaj opcję [licencjacka] dla pracy licencjackiej
% dodaj opcję [en] dla wersji angielskiej (mogą być obie: [licencjacka,en])
\documentclass[en]{pracamgr}

% Dane magistranta:
\autor{Katarzyna Kowalska}{371053}

% Dane magistrantów:
%\autor{Autor Zerowy}{342007}
%\autori{Autor Pierwszy}{342013}
%\autorii{Drugi Autor-Z-Rzędu}{231023}
%\autoriii{Trzeci z Autorów}{777321}
%\autoriv{Autor nr Cztery}{432145}
%\autorv{Autor nr Pięć}{342011}

\title{Approximation and Parametrized Algorithms for Geometric Set Cover}
\titlepl{Algorytmy parametryzowania i
trudność aproksymacji problemu pokrywania zbiorów na płaszczyźnie}

%\tytulang{An implementation of a difference blabalizer based on the theory of $\sigma$ -- $\rho$ phetors}

%kierunek: 
% - matematyka, informacyka, ...
% - Mathematics, Computer Science, ...
\kierunek{Computer Science}

% informatyka - nie okreslamy zakresu (opcja zakomentowana)
% matematyka - zakres moze pozostac nieokreslony,
% a jesli ma byc okreslony dla pracy mgr,
% to przyjmuje jedna z wartosci:
% {metod matematycznych w finansach}
% {metod matematycznych w ubezpieczeniach}
% {matematyki stosowanej}
% {nauczania matematyki}
% Dla pracy licencjackiej mamy natomiast
% mozliwosc wpisania takiej wartosci zakresu:
% {Jednoczesnych Studiow Ekonomiczno--Matematycznych}

% \zakres{Tu wpisac, jesli trzeba, jedna z opcji podanych wyzej}

% Praca wykonana pod kierunkiem:
% (podać tytuł/stopień imię i nazwisko opiekuna
% Instytut
% ew. Wydział ew. Uczelnia (jeżeli nie MIM UW))
\opiekun{dr Michał Pilipczuk\\
  Instytut Informatyki\\
  }

% miesiąc i~rok:
\date{June 2020}

%Podać dziedzinę wg klasyfikacji Socrates-Erasmus:
\dziedzina{ 
%11.0 Matematyka, Informatyka:\\ 
%11.1 Matematyka\\ 
%11.2 Statystyka\\ 
11.3 Informatyka\\ 
%11.4 Sztuczna inteligencja\\ 
%11.5 Nauki aktuarialne\\
%11.9 Inne nauki matematyczne i informatyczne
}

%Klasyfikacja tematyczna wedlug AMS (matematyka) lub ACM (informatyka)
\klasyfikacja{D. Software\\
  D.127. Blabalgorithms\\
  D.127.6. Numerical blabalysis}

% Słowa kluczowe:
\keywords{blabaliza różnicowa, fetory $\sigma$-$\rho$, fooizm,
  blarbarucja, blaba, fetoryka, baleronik}

% Tu jest dobre miejsce na Twoje własne makra i~środowiska:
\usepackage{amsfonts}
\usepackage{amsmath}
\newtheorem{defi}{Definition}[section]
\newtheorem{tw}{Theorem}[section]
\newtheorem{lemma}{Lemma}[section]

% koniec definicji

\begin{document}
\maketitle

%tu idzie streszczenie na strone poczatkowa
\begin{abstract}
  W~pracy przedstawiono prototypową implementację blabalizatora
  różnicowego bazującą na teorii fetorów $\sigma$-$\rho$ profesora
  Fifaka.  Wykorzystanie teorii Fifaka daje wreszcie możliwość
  efektywnego wykonania blabalizy numerycznej.  Fakt ten stanowi
  przełom technologiczny, którego konsekwencje trudno z~góry
  przewidzieć.
\end{abstract}

\tableofcontents
%\listoffigures
%\listoftables

\chapter{Introduction}
This is some very boring and really nothing on the topic introduction.
\chapter{Definitions}
Some definitions what geometric set cover is.
$\mathcal{P}$ -- set of objects, $\mathcal{C}$ -- set of points.
Choose $\mathcal{R} \subset \mathcal{P}$ such that
every point in $C$ is inside some $R \in \mathcal{R}$
and $|\mathcal{R}|$ is minimal.

In parametrized setting we only look among $|\mathcal{R}| \le k$.
In weighted settings there is some $f : \mathcal{P} -> \mathbb{R}$
and we minimize $\sum_{R \in \mathcal{R}} f(R)$.
 
\chapter{Geometric Set Cover with segments}

\chapter{Geometric Set Cover with lines}
\section{Lines parallel to one of the axis}
When $\mathcal{R}$ consists only of lines parallel to
one of the axis, the problem can be solved in
polynomial time.

We create bipartial graph $G$ with node for every line on the input
split into sets: $H$ -- horizontal lines and $V$ -- vertical lines.
If any two lines cover the same point from $\mathcal{C}$, then
we add edge between them.

Of course there will be no edges between nodes inside $H$,
because all of them are pararell and if they share 
one point, they are the same lines. Similar argument for $V$.
So the graph is bipartial.

Now Geometric Set Cover can be solved with Vertex Cover on graph $G$.
Since Vertex Cover (even in weighted setting) 
on bipartial graphs can be solved in polynomial time.

Short note for myself just to remember how to this in polynomial time:

Non-weighted setting - Konig theorem + max matching

Weighted setting - Min cut in graph of $\neg A$ or $\neg B$
(edges directed from $V$ to $H$)

\section{FPT for arbitrary lines}
You can find this is Platypus book.
We will show FPT kernel of size at most $k^2$.
For every line if there is more than $k$ points on it,
you have to take it. At the end, if there is more than $k^2$ lines,
return NO.

In weighted settings among the same lines with different weights
you leave the cheapest one and use the same algorithm.


\section{APX-completeness for arbitrary lines}
We will show reduction from Vertex Cover problem.
Let's take instance of Vertex Cover problem for graph $G$.
We will create set of $|V(G)|$ pairwise non-pararell lines
that any 3 of them don't cross in the same point.

Then for every edge in $(v, w) \in E(G)$
we put point on crossing of lines for vertices $v$ and $w$.
They are not pararell, so there exists exactly one such point
and any other line don't cover this point (any 3 of them don't
cross in the same point).
So this solving geometric set cover will choose
for every edge at least one vertex connected by this edge,
because we need to choose at least one of
lines corresponding to $v$ or $w$ to cover this point.

Vertex Cover for arbitrary graph is APX-complete,
so this problem in also APX-complete.

\section{2-approximation for arbitrary lines}
Vertex Cover has an easy 2-approximation algorithm,
but here very many lines can cross through
the same point, so we can do $d$-approximation,
where $d$ is the biggest number of lines crossing through the same point.
So for set where any 3 lines don't cross in the same point
it yields 2-approximation.

The problematic cases are where through all points
cross at least $k$ points and all lines have at least $k$ points on them.
It can be created by casting $k$-grid in $k$-D space on 2D space.

Greedy algorithm yields $\log |\mathcal{R}|$-approximation,
but I have example for this for bipartial graph and
reduction with taking all lines crossing through some point
(if there are no more than $k$) would solve this case.
So maybe it works.

Unfortunaly I haven't done this :(

I can link some papers telling it's hard to do.

\section{Connection with general set cover}
Of course every two lines have at most one common point,
so is every family of sets that have at most one point
in common equivalent to some geometric set cover with lines?

No, because of Desargues's theorem.
Have to write down exactly what configuration is banned.


\chapter{Geometric Set Cover with Weighted Polygons with $\delta$-expansion}
\section{Definitions}
This is very similar to every other paper about it.
\section{Problem Statement}
\section{APX-completness for rectangles with uniform weights (maybe segments)}
\section{$1+\epsilon$ approximation algorithm for weighted polygons of bounded thickness $\theta$}

\begin{defi}
  \textbf{Thickness} of the polygon
  is the ratio of the circumsribed circle's
  radius to the inscribed circle's radius.
\end{defi}

\begin{defi}[MWSCP]
	TODO: wstawić to jakoś wcześniej i inaczej
	Minimal Weight Set Cover for Polygons
\end{defi}

\begin{tw}[EPTAS for MWSCP with bounded thickness and $\delta$-expansion]
There is a randomized algorithm that given
a weighted family $\mathcal{P}$ of $n$ polygons
with thickness bouded by $\theta$
and set $\mathcal{C}$ of $m$ points
with total encoding
size of both sets $N$, and parameters $\delta$, $\epsilon$,
runs in time $f(\epsilon, \delta, \theta) \cdot (nN)^c$
for some computable functions $f$ and constant $c$,
and outputs a subfamily $\mathcal{S} \subseteq \mathcal{P}$
such that $\mathcal{S}^{\delta}$
covers the $\mathcal{C}$
and $w(\mathcal{S}) \le (1+\epsilon)OPT(\mathcal{P})$
with probability at least 1/2.
\end{tw}

\subsection{Sparsifying the family}
Intuitively, we will create a new input family
$\mathcal{P'}$ of polygons that 
can cover set of points $\mathcal{C}$
if and only if set $\mathcal{P}$
can cover set of $\mathcal{C}$
and OPT($\mathcal{P'}$)
is worse only by $\mathcal{O}(\epsilon)$-fraction of OPT($\mathcal{P}$).
The polygons in $\mathcal{P'}$ will be classified
into groups of similar size of edge of their
circumscribed squares.

Ogólnie wszystko tutaj będzie takie samo jak
w paperze, ale wstawimy stałą dla siatki 
$1/(\delta\theta\epsilon)$
zamiast $1/\delta\epsilon$.

$L = (1/\delta\theta\epsilon)^{\ell}$
for some $\ell = \mathcal{O}(N)$ -- limit for data.

Let's denote $d_i$ as a length of edge of the
circumscribed square on a polygon $P_i \in \mathcal{P}$.

\paragraph{Partition into layers}
Let's define a partition:
$$(\mathcal{P}_1, \mathcal{P}_2, \ldots, \mathcal{P_{\ell}})$$
of $\mathcal{P}$ and such reals $\nu_t, \mu_t$
for $t = 1, 2, \ldots, \ell$ with
the following properties satisfied for each $t \in \{1,2,\ldots, \ell\}$:
\begin{itemize}
\item $\nu_t \le d_i \le \mu_t$ for each $P_i \in \mathcal{P}_t$
\item $\nu_t = \mu_{t-1}$ (expect for $t=1$) and $\mu_t/\nu_t = (1/\delta\theta\epsilon)^{1/\epsilon}$
\item $\nu_1 \ge 1, \mu_\ell \ge L$, and all numbers $\nu_t$ and $\mu_t$ apart from $\nu_1$ are integers.
\end{itemize}

How to divide these polygons and choose numbers
is pretty straightforward,
but we also use some shifting parameter $0 \le b \le 1/\epsilon$
to be determined later.

$\nu_t = (1/\delta\theta\epsilon)^{t/\epsilon + b}$
$\mu_t = (1/\delta\theta\epsilon)^{(t+1)/\epsilon + b}$

\paragraph{Hierarchical grid structure}
Let $a \in \{1, \ldots, L - 1\}$ be an integer shift
parameter, to be determined later. Given $a$
we construct a hierarchy of grid lines in the plane.

For level $t$, define the \textit{level-t unit}
as $u_t = \delta \nu_t/(\theta 2\sqrt{2})$.
Note that $u_t$ is an integer.

We define a set of horiontal lines with $y$-coordinates
from the set:
$$a + b \cdot u_t : b \in \mathbb{Z}$$

Then for every polygon $P_i \in \mathcal{P}_t$
if the lines (horizontal or vertical)
from level $t+1$ cross the polygon $P_i$,
we split it according to lines
to at most 4 polygons with the
same weights and add these to $\mathcal{P'}$.
Otherwise $P_i \in \mathcal{P'}$.

\begin{lemma}
In polynomial time one can yield a family $\mathcal{P'}$
that satisfies $$OPT(\mathcal{P'}) \le (1+16\epsilon)OPT(\mathcal{P})$$
with probability at least 3/4.
Moreover one can construct the solution $S \subseteq \mathcal{P}$
back from the solution of $S' \subseteq \mathcal{P'}$
such that $w(S) \le w(S')$.
\end{lemma}

\textit{Sketch of proof}
If $\nu_t \le d_i \le \mu_t\epsilon$, then there is at most
$\epsilon$ probability that with random offset $a$,
the line will cut this polygon on the $t$-th level vertically.
Analogically for horizontal cuts.

If $\mu_t\epsilon < d_i < \nu_{t+1}$, then this situation happens
only for one $b$ in set $\{0, 1, 2, \ldots, 1\ \epsilon\}$.

Then for every polygon $P_i$ in optimal solution $OPT$, the expected value
of sum of weights for all polygons in $\mathcal{P'}$
corresponsing to the polygon $P_i$ is at most $4\epsilon$.

So with Markov inequality we can prove that
Pr($OPT(\mathcal{P'}) > (1+16\epsilon)OPT(\mathcal{P})) \le 1/4$

\paragraph{Extending polygons}
On every level $t$, for every $P_i \in \mathcal{P'}_t$,
we will create a new polygon $P_i'$ that
consists of every cell in hierarchical grid on level $t$,
that have non-empty intersection with $P_i$.

New polygon will fit inside $P_i$ shifted to every dimension by
$u_t\sqrt{2} = \delta \nu_t/(2\theta) \le \delta d_i/(2\theta)$.

The larger dimension is not extended more than by $\delta$:
$$2 \cdot \delta d_i/(2\theta) = \delta d_i/\theta \le \delta d_i$$

The shorter dimension is at most $d_i/\theta$,
so it also wouldn't be extended by more than $\delta$.



\subsection{Dynamic programming}






\chapter{Conclusions}

\begin{thebibliography}{99}
\addcontentsline{toc}{chapter}{Bibliografia}

\bibitem[Bea65]{beaman} Juliusz Beaman, \textit{Morbidity of the Jolly
    function}, Mathematica Absurdica, 117 (1965) 338--9.

\bibitem[Blar16]{eb1} Elizjusz Blarbarucki, \textit{O pewnych
    aspektach pewnych aspektów}, Astrolog Polski, Zeszyt 16, Warszawa
  1916.

\bibitem[Fif00]{ffgg} Filigran Fifak, Gizbert Gryzogrzechotalski,
  \textit{O blabalii fetorycznej}, Materiały Konferencji Euroblabal
  2000.

\bibitem[Fif01]{ff-sr} Filigran Fifak, \textit{O fetorach
    $\sigma$-$\rho$}, Acta Fetorica, 2001.

\bibitem[Głomb04]{grglo} Gryzybór Głombaski, \textit{Parazytonikacja
    blabiczna fetorów --- nowa teoria wszystkiego}, Warszawa 1904.

\bibitem[Hopp96]{hopp} Claude Hopper, \textit{On some $\Pi$-hedral
    surfaces in quasi-quasi space}, Omnius University Press, 1996.

\bibitem[Leuk00]{leuk} Lechoslav Leukocyt, \textit{Oval mappings ab ovo},
  Materiały Białostockiej Konferencji Hodowców Drobiu, 2000.

\bibitem[Rozk93]{JR} Josip A.~Rozkosza, \textit{O pewnych własnościach
    pewnych funkcji}, Północnopomorski Dziennik Matematyczny 63491
  (1993).

\bibitem[Spy59]{spyrpt} Mrowclaw Spyrpt, \textit{A matrix is a matrix
    is a matrix}, Mat. Zburp., 91 (1959) 28--35.

\bibitem[Sri64]{srinis} Rajagopalachari Sriniswamiramanathan,
  \textit{Some expansions on the Flausgloten Theorem on locally
    congested lutches}, J. Math.  Soc., North Bombay, 13 (1964) 72--6.

\bibitem[Whi25]{russell} Alfred N. Whitehead, Bertrand Russell,
  \textit{Principia Mathematica}, Cambridge University Press, 1925.

\bibitem[Zen69]{heu} Zenon Zenon, \textit{Użyteczne heurystyki
    w~blabalizie}, Młody Technik, nr~11, 1969.

\end{thebibliography}

\end{document}


%%% Local Variables:
%%% mode: latex
%%% TeX-master: t
%%% coding: latin-2
%%% End:
